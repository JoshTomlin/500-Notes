\documentclass[a4paper]{JoshCards}

\title{Memorable 500 Hands}
\author{Josh Tomlin}
\pagestyle{fancy}
\date{}



\begin{document}

\maketitle

Here is a collection of humorous and interesting hands of 500 that I have played over the years. Most of them will be with the recurring cast of myself, Bazli, Riley and John, usually playing at the Woden Southern Cross Club. I do my best to remember the cards and play, however I cannot always remember the exact shape or spot cards of every hand. I can get close with my own hand, but doing it for everyone at the table is quite the challenge. So, I have tried my best to be as accurate as possible, and when unsure I have chosen cards that make the hands make as much sense as possible.

\subsection*{14th July 2020 - WSC}

Another evening at the Woden Southern Cross Club playing cards over dinner. I am sitting South partnered with Bazli, Riley is on my left and John is on my right.

\subsection*{A Misere Beer}

\setdefaults{err=off}
\gamefont{\larger}
\boardnr{1}
\northhand{AKT}{K976}{Q5}{T}
\southhand{QJ95}{}{K87}{AK8}
\easthand{-}{AJT54}{JT964}{-}
\leftupper{}%
{}{}
\rightupper{\contract:
OMis.}{\declarer: \east}{}
\rightlower{\lead: \Di 6}{}{}
\showAll*

John reaches what he thinks to be a solid Open Misere. As he is gloating about how great his hand is after the kitty, I think about what discard will surprise him the most on a heart lead. I decide on the \D K. In my head, I'm hoping the play goes \H5 lead, \D K from me and \He4 from Bazli for a hilarious instant loss. I can imagine the uproar now: ``How am I so unlucky! This distribution is beyond stupid". Sadly, he leads the \D6 instead. I almost put the \D7 on it, to tell Bazli about my \D8, but I figure this is too risky and a play I would only make at Closed Misere to discourage the suit. Instead, I play the \D K, coincidentally the same card I had originally planned.Under my \D K, Bazli drops the \D Q. John puts down a solid dummy and sits there, very proud of his hand. 

I count the diamonds and realise only the \D A and \D5 are missing. Bazli cannot have the \D A because he would have played it at trick one. Thus I know he started with either singleton \D Q or \D Q5 doubleton. So I lay down the \D8, Bazli follows with the \D5 and John plays the \D4. I slam the \D7 on the table, exclaim ``Beer!", and put my hand away. John, confused, asks what Diamond Bazli is going to play, to which I respond ``He has no Diamonds". Sure enough, Riley puts down his hand with the singleton \D A. 

To add insult to injury, Bazli shows me his hand and says he has John beat in hearts too. On this layout, neither of the two `solid' suits John claimed to have were safe at all. So much for a great Misere hand, and out the bog they went.


\subsection*{Seven Spades is Forcing!}

\todo[inline]{Aligning formatting is all off}

\begin{center}
    
\setdefaults{err=off}
\gamefont{\larger}
\boardnr{1}
\northhand{YAQ}{J6}{QJ9}{Q5}
\easthand{5}{9754}{54}{J87}
\southhand{T987}{AQ}{86}{K6}
\westhand{J6}{T8}{KT7}{AT9}
\rightupper{}{}{}
\leftupper{\contract:
7\Sp}{\declarer: \east}{}
\rightlower{\kitty: \D A \H K \S K}{}{}
\showAll*

\begin{bidding}% default= [c]
       &   & 6S* & p \\
    6C & p & 7S** & - \\
    p*** & - &\\
\end{bidding}

* - Wants Misere\\
** - Really wants Misere\\
*** - Oh no...
\end{center}

Thought about bidding 6S before John did it.
John admitted he wanted Misere after the auction ended.
Bazli said to Riley "Come on, of course 7S is forcing!!".

John proceeds to put the 5S in the kitty and leads the DA. then plays C7, Riley winning the A. Bazli joking we would have won misere on first trick, John claiming he would have put 3 trumps in the kitty. 3 trumps and the 7C?? I ask him.

I joke about how long we are going to delay playing trumps, John looks at me funny. I win my QH and play 7S to Bazli's Q and K. John plays a heart, I win and play the 10S to the J, Jo and small heart. We all laugh, then John picks up his card and replaces it with the 8C. Bazli goes to lead a diamond, John plays 8C and tries to claim that he won the trick. We laugh and say no, clubs aren't trump spades are. he thought clubs were trumps the whole time. Plays JC later, which is a penalty card and gets discounted. Bazli leads the QD, then John tells us he was playing the contract with only two trumps instead of three, and that one was in the kitty. We're all shocked. I don't even know what to do with my hand. Eventually I conclude that Riley started with Jx of spades, John had 3 but put one in the kitty, therefore Bazli had 3. So I could ruff the QD safely and claim the rest of the tricks by cashing the KC.

\setdefaults{err=off}
\gamefont{\larger}
\boardnr{1}
\northhand{YAQ}{J6}{QJ9}{Q5}
\easthand{K}{K9754}{A}{J87}
\southhand{T987}{AQ}{86}{K6}
\westhand{J6}{T8}{KT7}{AT9}
\leftupper{}%
{}{}
\rightupper{\contract:
7\Sp}{\declarer: \east}{}
\rightlower{\lead: \Di A}{Discard Kitty: \He 5 \Di 4 \Sp 5}{}
\showAll*



\subsection*{1st Aug, 2020}

Playing tonight partnered with Shaanan against Andrew and Kelly at a new venue, Andrew and Kelly's house. A full evening of 500, only taking a break to eat dinner.

\subsection*{Guiding Partner}

\setdefaults{err=off}
\gamefont{\larger}
\boardnr{8}
\northhand{}{}{}{}
\southhand{KT}{J98}{J8}{AJ7}
\westhand{AJ654}{75}{75}{6}
\easthand{}{}{}{}
\leftupper{}%
{}{}
\leftupper{\contract:
OMis.}{\declarer: \east}{\lead: \Cl 6}
\rightupper{}{}{}
\rightlower{}{}{}
\showAll*

Andrew, sitting East, reaches Open Misere and leads the \C 6. Shaanan plays the \C T and I play the \C A.

At the table I quickly played the \S K and then the \S10, readying myself to be short suited, Shaanan following with the \S 9 and \S 7. I led the \D J, hoping that Shaanan had the \D 4 and an extra spade to short suit me in diamonds. Shaanan left me on lead, so I tried a heart to give her the lead. Turns out, she only had the \H 4 and the Joker (her only spade left), which was not enough to beat the contract today.

This hand is interesting from our point of view. There are two clear equally good weaknesses to attack, but we can only ever hope win in diamonds. My play of the \D J was to try to tell this to partner, but I think we should make this play before playing spades. The only thing that matters in this hand is telling partner what suit to attack. Therefore, we should give this information to partner as early as possible, and leading the \D J first does that. There is no reason to shell our spades early, partner will know that this is the obvious place to short suit me before trying a low heart.

Suppose that our hearts and diamonds were swapped around. Then we can make the exact same play in hearts. The danger of playing spades first is that they may be blocked, leaving partner on lead without the information they need to beat the contract. They can try hearts or diamonds themselves, but we've lost my opportunity to tell partner what to do. If they have only one (or none!) of the red fours, what we do doesn't matter because parter will take the reigns (hopefully they have more spades to short suit you, and tell you which suit to discard).

Telling partner what to attack early opens up a powerful negative inference from partner's point of view. If partner knows that we will tell them which suit to attack as early as possible, then playing the spades first tells partner that we have no preference (e.g we are 2-2 in the reds). Now partner can take the lead in spades and tell us which suit to discard. If they have the \H 4, they should lead a high heart before playing another spade (on which we will discard a heart) or similarly if they have the \D 4. Even if they don't have a high heart to go along with their \H 4, they can lead two high diamonds to eliminate that option from partner's radar, and making it clear that we should discard a diamond. If partner really trusts us and doesn't have a lot of high cards, they may even be able to lead a low diamond, helping us unblock the suit.


\subsection*{Eight Hearts!}

\setdefaults{err=off}
\gamefont{\larger}
\boardnr{1}
\northhand{}{}{}{}
\southhand{5}{T7654}{J75}{6}
\westhand{}{}{}{}
\easthand{}{}{}{}
\leftupper{}%
{}{}
\leftupper{}{}{}
\rightupper{}{}{}
\rightlower{}{}{}
\showAll*

I pick up an interesting hand here. Lots of hearts, but the suit quality isn't great. This looks a good hand to chance Open Misere on, even though the \D J75 weakness could be quite bad. Not sure how to bid it, but thankfully there are two people to bid in front of me.

While I'm thinking about what to do with this hand, Shaanan opens 8 Hearts. I almost jump out of my chair, overwhelmed with shock and stare at her, to which she quickly corrects herself. ``Six Hearts! I mean Six Hearts". Everyone's laughing now, but I'm obviously still confused. I have no idea what she has here, a monster with hearts? Do I bid 10 hearts? Was it just a slip of the tongue and should I bid 7 or 8 hearts? What are my chances at open misere now? The diamonds are looking worse... 

Kelly bids 7 spades while I'm staring at my hand confused. I'm trying to pass it off as haha, you shocked me by opening 8 hearts which you never do, but the one time this happens I happen to have 6 card support?? What could possibly be going on.

Eventually, I decide to bid a mere 7 hearts, trying to get to misere or a good heart contract. Maybe Shaanan will raise me to 10 if she has a monster. This gets passed out, and I pick up the kitty, quite disappointed to only be in a 7 bid. In the kitty, I happen to find

\begin{center}
\suit[\He]{AQ9}
\end{center}

Now I am at a loss for words. Three hearts in the kitty!?!?! I now have 9 trumps! How is this possible? Keeping the beer card for good measure, my hand is now

\setdefaults{err=off}
\gamefont{\larger}
\boardnr{1}
\northhand{}{}{}{}
\southhand{}{bAQT97654}{7}{}
\westhand{}{}{}{}
\easthand{}{}{}{}
\leftupper{}%
{}{}
\leftupper{}{}{}
\rightupper{}{}{}
\rightlower{}{}{}
\showAll*

What on earth could Shaanan have if I have all the trumps. I'm starting to suspect that she has no trumps at all! I lead the \H 4 because why not. Andrew discards a club, Shaanan wins the Joker and Kelly also discards a club. Now I'm laughing because we have a 13 card fit. Shaanan plays the \H K and I duck to keep her on lead. She lays down the \D A and I quickly show everyone my outrageous hand and claim.

Shaanan's heart suit was Joker, Bower, King, Eight. Turns out, she did have her six heart bid. She tells us that the reason she accidentally said Eight Hearts was because she was looking at the \H 8. Surprisingly, that possibility never occurred to me. I wonder what stars had to align to have all those events come together.


\subsection*{0-0-5 trumps?}


\setdefaults{err=off}
\gamefont{\larger}
\boardnr{1}
\northhand{}{}{}{}
\southhand{Ax}{A}{YBAQ87}{x}
\westhand{}{}{}{}
\easthand{}{}{}{}
\leftupper{}%
{}{}
\leftupper{}{}{\contract: 10\D}
\rightupper{}{}{}
\rightupper{}{}{\kitty: \H J \D 9 \C x}
\showAll*

It's late in the third game of the night, our side doesn't have a win yet and the points are around 440-300. I don't remember all of the auction, but Shaanan had made at least one positive bid. I tried to steal the auction low, hopefully 8\D. Alas, Kelly tries Open Misere and I quickly bid 10\D over it. A stretch, but I'm in the mood to try my chances.

The kitty is something from my wildest dreams. I add the two diamonds to my hand and now I have 

\par\noindent
\hand*!{A}{A}{YBbAQ987}{}

I'm laughing about how great my hand is after the amazing kitty, while Shaanan leaves the table because she's been organising dessert. I almost show my hand to everyone to claim, but I take a second to think about what could possibly go wrong. I have eight trumps to the top four, partner showing up with no diamonds and a 5-0 trump split is the only thing that could hurt me. Okay, I'll play it out. 

Before leading I yell out to Shaanan - ``Hey Shaanan, do you have any diamonds?''. She's in the kitchen so can't look at her hand, but says ``Ahh, I don't think so''. Oh no. The joy has drained out of my face. I tell Andrew that I'm just praying that no one has five trumps. I lead the \D A while still waiting for Shaanan. Andrew discards and starts laughing. Oh no again. I'm frantically thinking about any sort of play I can make to pick up 5 trumps to my right. A simple finesse will do, but Shaanan has no trumps. A trump coup? I can't short suit myself, let alone get any entries to partner's hand. There's nothing I can do, so I just have to sit tight, anxiously waiting for Shaanan to come back and show me her diamond void.

Andrew is laughing and having a great time over my suffering, Kelly is slyly staying quiet, and I'm nervously waiting for Shaanan. Shaanan tells me to just look at her hand, but I tell her I can't do that. If only this was bridge and I could put dummy down and get the bad news out of the way quickly. After what feels likes an eternity, Shaanan finally returns. She picks up her hand, says ``Oh'', and plays a diamond. What a huge sigh of relief for me. Turns out, Shaanan had three diamonds and Kelly two. Kelly says she knew she didn't have five diamonds and I had nothing to worry about, but she thought it was funny to keep quiet. At least I got to relax and enjoy my sticky date pudding after.


\subsection*{Power of Position}


\setdefaults{err=off}
\gamefont{\larger}
\boardnr{1}
\northhand{AKxx}{QJ6}{A}{T7}
\westhand{xx}{AK97}{xxxx}{}
\southhand{x}{x}{K86}{YQ985}
\easthand{6}{T84}{J7}{JAK6} 
\leftupper{}%
{}{}
\rightupper{\contract:
8\Cl}{\declarer: \south}{}
\rightlower{\kitty: \S J \D 5 \S x}{}{}
\showAll*


\setdefaults{err=off}
\gamefont{\larger}
\boardnr{1}
\northhand{AK}{J6}{A}{T7}
\westhand{K}{K9754}{A}{}
\southhand{}{}{K865}{YbQ985}
\easthand{J6}{T8}{KT7}{JAK6}
\leftupper{}%
{}{}
\rightupper{\contract:
8\Cl}{\declarer: \south}{}
\rightlower{\lead: \Di A}{Kitty: \Sp J \Di x \Sp 5}{}
\showAll*






%\begin{bidding}% default= [c]
 %   1\Cl\announce & 1D & 1H & 1S\alert \\
  %  1N & X & p & p \\
   % R & P \\
    %\end{bidding}




\setdefaults{err=off}
\gamefont{\larger}
\boardnr{1}
\northhand{YBbA}{QJ6}{A}{T7}
\westhand{}{AK97}{xxxx}{}
\southhand{9875}{x}{AK86}{YQ985}
\easthand{6}{T84}{J7}{JAK6} 
\leftupper{}%
{}{}
\rightupper{\contract:
8\Cl}{\declarer: \south}{}
\rightlower{\kitty: \S J \D 5 \S x}{}{}
\showAll*





\newpage
\subsection*{Suit Preference?}

Playing at the WSC on Sunday the 30th August 2020 with the OG banter crew. I'm partnered with Bazli, and in a tight game we're playing 8C with these hands (post kitty). Riley (West) probably bid diamonds at some point.

\setdefaults{err=off}
\gamefont{\larger}
\boardnr{1}
\northhand{A96}{}{A98}{YQT5}
\westhand{K8}{AJ}{KJ4}{bK9}
\southhand{Q5}{T65}{T}{BA86}
\easthand{T7}{K84}{Q765}{7} 
\leftupper{}%
{}{}
\rightupper{}{}{}
\rightlower{}{}{}
\showAll*

Bazli starts by leading the Joker and a small trump to me. I follow first with the \C6 (giving count) and then winning the \C J. John (West) shows out on the second round, and Riley drops the \C K. 

I'm quite confident that Riley either started with King doubleton or bKx in trumps. He could possibly have KQx, but he doesn't often make plays like that. It is now \textit{crucial} that I a stop drawing trumps because there is nothing to be gained. If Riley has the \S J, it is going to win anyway, so my goal is to try and win our trump tricks separately. So I quickly play a diamond, hoping to be able to ruff some of Bazli's diamond losers.

I'm pretty happy when Bazli wins the \D A. He thinks for a little, then returns the \D 9. I happily ruff, and am immediately thinking about whether Bazli has given me a suit preference signal. This would be a standard play in bridge (only necessary on defence though) and we have discussed this idea before. The \D 9 is suspiciously high, so I figure I'll play back a spade and hope that Bazli is a legend of a partner.

I'm loving life when Bazli wins the \S A. He plays back the \D 8 and I ruff with the \C A, noting the \D K falling to my left. The four card ending is:

\setdefaults{err=off}
\gamefont{\larger}
\boardnr{1}
\northhand{96}{}{}{T5}
\westhand{K}{AJ}{}{b}
\southhand{Q}{T65}{}{}
\easthand{T}{K8}{Q}{} 
\leftupper{}%
{}{}
\rightupper{}{}{}
\rightlower{}{}{}
\showAll*

I'm now not sure what to do. I know Bazli has two trumps (or three, but in that case we've already won), so I try to set up any spades he has by banging down the \S Q. This gets covered by Riley's \S K and pins John's \S T. Riley cashes the \S J to draw one of Bazli's trumps, but now there is nothing he can do. Bazli simply has a trump and a winning \S 9, and says he thinks he can claim (a little unsure whether his \S 9 is actually good).

I ask Bazli after the hand whether he gave me a suit preference signal, and he said he didn't think about which diamond he played. Also, in my head a high diamond means a spade, but that's working off the bridge ranking of the suits (the 500 ranking is quite different, so we would have to agree on how we play our suit preference signals). So I realised that that would have made any signal unclear anyway. 

It's kind of lucky for us that I got it right this time, even without the help of a signal. The only way to make the hand is to ruff two diamonds in my hand, and it's crucial that I lead back a spade at trick 5 (to get my second ruff) rather than a heart because I can't afford to tap Bazli. If his trumps are reduced, then when Riley gets in with a spade he can draw Bazli's last trump and start running hearts, making us give up and cry. This hand illustrates how important suit preference signals can be in tightrope contracts such as this one. But hey, sometimes it's better to be lucky than good.



\subsection*{Three, three, 3-3}

\setdefaults{err=off}
\gamefont{\larger}
\boardnr{1}
\northhand{AK75}{}{bxxx}{Kx}
\westhand{Jx}{x}{Yxx}{Jxxx}
\southhand{Qx}{Axx}{BKx}{xx}
\easthand{Tx}{Kxxx}{Axx}{x} 
\leftupper{}%
{}{}
\rightupper{\contract:
8\Cl}{\declarer: \south}{}
\rightlower{\kitty: \H xxx}{}{}
\showAll*


\newpage
\subsection*{Partner Has You Covered}

Playing at WSC with the OG crew on a Wednesday night, 16/9/20. I'm in 4th seat and pick up this relatively boring hand. 
\begin{center}
\par\noindent
\hand{}{AT9764}{KT}{Q9}
\end{center}
I note that the heart pips are amazing for Misere, but no way am I going to get there with the other serious weaknesses.

Riley opens the bidding on my left with Six Spades. Bazli bids 6 Clubs and John jumps to Nine Spades. We're all shocked over John's bid, especially given his reputation (maybe he wants open misere?). I'm having no part of it and quickly pass. Riley passes and Bazli takes a while to think. Eventually he says ``yeah I'll try it" - Open Misere. 

\begin{center}
\begin{bidding}% default= [c]
  6S & 6C & 9S & p \\
  p & OM & p \\
\end{bidding}
\end{center}


After John passes and I ask him ``Why did you jump to Nine Spades??". He says he thinks he was making it, and excitedly shows me his hand.

\begin{center}
    \par\noindent
    \hand{YAQ96}{}{AQJ4}{J}
\end{center}

It's a pretty great hand, 6-4 with just two losers, and Riley probably has the \S J from the bidding. The \D K is offside, so at least they wouldn't make 10. Bazli is taking a while with the kitty. Eventually, he huffs with disappointment and worry about how bad his hand is and leads the \C 8. I'm worried for a second thinking John has a Club void and Bazli is going to get embarrassed on the first round, but then I remember he has stiff \C J. Thinking the \C 8 was a weakness, I thought Bazli had been quite lucky. Riley follows with the \C A, and Bazli puts down the following dummy.

\begin{center}
    \par\noindent
    \hand{}{KJ85}{}{KT765}
\end{center}

We look at it for a couple seconds. His Clubs look amazing, but his Hearts look quite bad. Then I look at my hand and realise that I have literally every card he is missing baring the \H Q. I break the silence with a loud ``Welp'', and throw down my six hearts. John and Riley are in disbelief and completely outraged. The full deal is:

\setdefaults{err=off}
\gamefont{\larger}
\boardnr{1}
\northhand{T}{KJ85}{}{K8765}
\easthand{YAQ96}{}{AQJ4}{J}
\southhand{}{AT9764}{KT}{Q9}
\westhand{JK54}{}{98765}{A} 
\leftupper{}%
{}{}
\rightupper{\contract:
8\Cl}{\declarer: \south}{}
\rightlower{}{}{\kitty: \H Q \C T \S 8}
\showAll*

Bazli shows us his hand and the kitty he got. John is going on and on about how ridiculously lucky Bazli keeps getting. ``Your partner has 6 cards covering you, and you got a club in the kitty to get rid of your spade!''. I add that the \H Q would have also been safe to keep. This cracks up Bazli but does temper John. Bazli justifies his gamble by telling John that he just ``listened to the bidding''. John is now seriously fuming.

I will admit, it's not everyday that you have a 12 card fit with partner and the opponents have a double void in your weak suit. Good thing he didn't lead it!


\newpage
\subsection*{Pin the 5!}

Playing at WSC with the OG crew on a Sunday night, 25/10/20, with the NRL grand final on in the background. I'm in 4th seat and pick up this monster. 
\begin{center}
\par\noindent
\hand{65}{J5}{8}{AT876}
\end{center}
The bidding goes:
\begin{center}
\begin{bidding}% default= [c]
  6D & 6H & 7S & Mis \\
  8D & 9D* & p & OM \\
  P\\
\end{bidding}

* - Showing a diamond control, denying a spade and club control (therefore implying a heart control).
\end{center}

Open misere on this hand feels quite pushy and not a place I want to be, but partner has put us here so we have to live with it. I'm confident that Bazli's 9\D bid has denied the \C5, which means my club suit is a huge weakness. Shockingly, the kitty is \C J9, \D 5. Two clubs and a small diamond? Are you kidding me! Adding these to my hand I hold:
\begin{center}
    \par\noindent
    \hand{65}{}{5}{AJT9876}
\end{center}
Now, what do you lead?

I think a club lead is absolutely necessary and clear from the bidding. The defenders hold the \C5, your only weakness, and are good enough to find a way to short suit their partner. Riley has shown long diamonds, John has shown long spades, and I am certainly going to lose the discarding race if either of them decide to run these suits. By leading a club, I can pin the \C5 whenever it is singleton, which is actually rather likely considering there are only three clubs missing (\C KQ5).

Now the question becomes - what club do I lead? I decided I was leading a club early (probably before I picked up the kitty) so I spent the rest of my thinking time deciding how to con Riley into playing the \C5 from \C K5 or \C Q5. I ended up leading the \C J (my highest) but I think this is too obvious and a much more subtle lead would be more convincing. Leading the Jack just gives the show away, advertising to the world that I have a long solid clubs. If I lead the \C7 or \C8, it takes a good player to even consider holding up the \C5, and even then it's unclear what to do. I think that, given my clubs were undisclosed from the bidding, the \C7 is the most likely to pull one over on leftie. There are chances that it's a singleton \C 7 and partner will never forgive you for not playing the 5 when they have the 6! They also may read into the bidding to know that Bazli cannot have the \C6, so maybe their partner does?

As it happens, I lead that \C J and Riley instantly played the \C5! Turns out he had singleton 5 and my club lead was necessary and the winning play! Feels good to get that 500 points when a more normal diamond or spade lead leaves you dead in the water. When John plays the \C K I quickly check that Bazli has the \D 4 and claim. Turns out that today, any club lead would have worked (except the \C A of course!).






\newpage
\subsection*{A Critical Psych}

Same night as above and the tension is high. After going up 5-4, Riley tried to call the night but I convinced everyone to run it back one more time. The score is 330 to -300 with our side in the bog. In first seat, you hold
\begin{center}
\par\noindent
\hand{Q9765}{K}{AJ}{86}
\end{center}
Given the score vulnerability, I have to open this hand to keep myself in the bidding, possibly planning to try a slim open misere or hopefully push them too high. I briefly consider opening 7\S as a pre-empt, but this doesn't seem to work in 500 (they just bid over you usually). So I open 6S. The bidding continues
\begin{center}
\begin{bidding}% default= [c]
       &    & & 6S \\
    6C & 6H & 6N & 7S \\
    8C & p & p & ?\\
\end{bidding}
\end{center}
My hand is of course outrageously weak for a 7\S  bid, but we're desperate and I need to make something happen. Before making his last bid, Riley looked at the score, and then jumped to 8\C  (of course realising that 260 would put them over the line while 160 will not). I'm in a bit of a pickle in my spot considering that 8\C seems very likely to make (I have at most one defensive trick) while no other bid seems even remotely attractive (maybe open misere on a kitty sent from heaven). I decide to make the tactical bid of 8\H(!?) - supporting partner, hoping for some trumps in the kitty if I have to play it, but hoping more that I can convince Riley to bid at the 9 level. 

Up until this hand, Riley had been riding a wave of unbelievable luck with the kitty, and just the good hands he was being dealt overall. He ums and ahs for a while, but I know this man is a punter for style points and that he is very likely to go for the glory. He bids $9\Cl$ and I pass faster than I have ever passed before. My job here is done, now I need only to hope for one trick from partner. 

Riley begins by leading the Joker and then the \S J. Bazli follows once (I miss his discard in the stress of the whole situation) and John follows small and then \C A. This leads me to believe the John has \C JAx or \C Ax, with Riley starting with 6 or 7 trumps. He then leads a low diamond to John's \D Q and my \D A. While it's great for me that my diamond cards are useful, I now become very worried that Riley's plan is to play the \D K at some point, dropping my stiff Jack and setting up his suit. 

I can't afford to blow a trick here, so I lead the \H K in the hopes that either Bazli has the \H A or Riley will ruff it. Riley ruffs and then leads the \C J. I now know that he started with 7 trumps and his distribution is most likely 0=0=3=7 (I don't think he is leading away from Kx in diamonds). Now I'm panicking that his diamond suit is running. My only hope is that partner holds Txx or better (maybe 9xx if John has QT tight or pitches the T) but in 500, no one is counting and partner will certainly discard his diamonds at some point. How can I convince partner to guard diamonds?

On Riley's lead of the \C J, I discard the \D J!! When partner sees this happen, alarm bells should be ringing. I'm essentially screaming ``Hey partner! I cannot stop diamonds, please help!". This should be a clear signal because Riley has already attacked diamonds, and I would never discard here from \D Jxx. I hope that this message gets through to Bazli because otherwise he certainly will discard his diamonds without a second thought. On the other hand, this discard can never cost. If Riley's Diamonds are \D KTx, all hope was lost anyway when his partner showed up with the queen. If partner has the \D K, we will defeat the contract anyway.

On the next round I see John discard the \D T. Now I think it's over because partner certainly isn't holding onto 9xx in diamonds. However, after running trumps Riley plays a low diamond and Bazli wins the \D K! We take down the contract and not only remain alive, but have put ourselves back into this match!

It turns out that on this hand, nothing I do matters since Bazli has the \D K, but it's good defence to find chances in the worst of cases. I tell Riley that I only had two hearts for my $8\He$ bid and Bazli informs us that he had four hearts to the Q. This means Riley could have pitched a diamond on heart lead and let his partner in (risky since he has already lost a trick). The problem with this line is he is still relying on his partner having another trick for his second diamond loser, so he might need to have the \D K anyway. It's possible John should have the \S A too for his bid, although he rarely does have three Aces. This dramatic change in scores and lucky really changed the dynamic and tone of the game, and afterwards the cards seemed to fall very in our favour. This is a real example of turning a game around when your back is up against the wall and creating your own luck.


\newpage
\subsection*{Passive Defence}

Playing on a Wednesday night before christmas (23/12/20) at Jerrabomberra Avenue with Andrew and Kelly. Mixing things up a bit by partnering Andrew and having Shaanan on my right. After being dealt an outrageous diamond suit and making 9NT, we are up 420 to 0. In first seat you pick up:
\begin{center}
    \par\noindent
    \hand{}{9xx}{Ax}{YbKT9}
\end{center}
I start with $6\Cl$, Kelly bids $6\Di$ on my left and Andrew bids $7\Sp$. I decide to stay in the bidding with $7\Cl$, my hand should be okay opposite a possible bower from my partner. Kelly asks what the scores are and then bids $7\Di$ over me with Andrew passing. I get a read that Kelly isn't interested in letting me play clubs given the current score, and my hand is going to be quite good defensively against a diamond contract. So I leverage our score advantage with a tactical `bluff bid' of $8\Cl$. This will obviously be a terrible contract if I get left it in, but I'm quite sure I won't be, and even so maybe I'll get lucky with partner or the kitty. Kelly obliges by bidding $8\Di$ and I quickly pass, happy with the position I'm in to possibly put an even larger disparity between the scores.
\begin{center}
    \begin{bidding}
           &    & & 6C \\
        6D & 7S & p & 7C \\
        7D & p & p & 8C\\
        8D & P\\
    \end{bidding}
\end{center}
Now defending diamonds, I resort my hand to look like this:
\begin{center}
    \par\noindent
    \hand{J}{9xx}{YAx}{KT9.}
\end{center}
Kelly leads the \D J and I win with the Joker. I play my stiff spade towards partners suit, importantly, through declarer. Kelly ruffs and says something like 'thank you'. She then plays \H J and another diamond. I win the ace while Andrew and Shaanan show out, leaving me with 
\begin{center}
    \par\noindent
    \hand{}{9xx}{}{KT9.}
\end{center}
What now?

Counting declarer's hand we know that she started with 6 hearts and a void spade, leaving four other cards. We've already won two tricks and only need one more, so my only goal here is to not give away a trick. This is why I played a spade the first time I was in, thinking it was probably safe playing through declarer towards my partner's suit. Her ruffing it was fine by me because it did not give away a trick, and ducking the spade to partner's hoped for Ace would be quite risky early, especially considering Andrew's spade bid. 

Not wanting to risk blowing the club suit, I lead a top heart, the most passive/safest thing I could think of. Kelly looked excited to win the \H A and cash her \C K, but then looked rather disappointed and played a club. I won my King and that was it, down 1. I didn't see all the hands but I think they were something along the lines of:

\setdefaults{err=off}
\gamefont{\larger}
\boardnr{1}
\northhand{AKxx}{Kxx}{xx}{x}
\easthand{xxx}{Qx}{xx}{Jxx}
\southhand{J}{9xx}{YAx}{KT9}
\westhand{}{A}{BbKQxx}{AQx}
\leftupper{}%
{}{}
\rightupper{\contract:
8\Di}{\declarer: \west}{}
\rightlower{}{}{}
\showAll*

It's possible that Andrew held the \C J (especially for his $7 \Sp$ bid) but if Shaanan has it then it illustrates the importance of a passive defence. If I naively play back a club at trick 5 (after getting in with the \D A) then a good declarer can duck it around to their partner's \C J, hoping that you've underlead your \C K. A good partner will then play a Club back through me and pick up the whole suit, making eight tricks! Ducking the Club return may look risky for declarer, but at this point it looks like the only hope is for partner to have something in clubs anyway so you're just giving yourself a bonus chance that the lead is away from the \C K. 

On the flip side, a heart lead can only cost if Andrew has the \C A and we need to cash it immediately. For example, give Kelly a stiff club, Axx in hearts and give Shaanan the \H K and some winners in spades or hearts. Then Kelly can get to Shaanan's hand through hearts and pitch clubs on Shaanan's winners. That's a lot of ifs and this situation is especially unlikely given the bidding and play. It's also difficult to construct a hand where returning a heart blows the heart suit since any of partner's honours will be sitting over declarer (the strong hand). In this case, partner has the \H K which is perfect sitting over the \H A (especially if it wasn't a stiff Ace).


\subsection*{Beer?}

This next hand (same night as above) is a peculiar one. It still remains a mystery what happened in the ending.

You're second to bid and get dealt this pile of rubbish:
\begin{center}
    \par\noindent
    \hand{T5}{QT8}{94}{987.}
\end{center}
The auction goes something along the lines of 
\begin{center}
    \begin{bidding}
           &   & 6H & p \\
        7C & p & 7H & P \\
    \end{bidding}
\end{center}

Shaanan leads a low heart. I had recently been discussing count signals in trumps with Andrew, so I play the \H Q to show 3 hearts. Somehow, this held the trick! Really not sure of what to do, I switch to a spade with the thought of getting a ruff. It goes \S T, \S J, \S 9, \S Q. Before playing the \S Q, Shaanan hesitated and then said sorry to Kelly. I didn't realise this at the time but this is a strange trick and actually contains a lot of information. You can be quite sure that Shaanan has the \S A because 500 players love to win their Aces and the first possible opportunity. So it looks like Shaanan has \S AQ tight (or possibly \S AKQ). Further, partner can't have the \S K so probably Kelly has it.

Shaanan cashes the \S A, which is unnecessary and sets me up for a spade ruff (possibly an uppercut!). She then leads another low trump and this goes around to Andrew's \H J, Kelly showing out. It looks like Andrew has the Joker (Shaanan hasn't led it yet) and Shaanan has six hearts. I was hoping Andrew would continue with a spade, giving me a ruff or trump promo, but he played a club instead. Shaanan ruffed and led a final low trump, Andrew winning the Joker. He plays a low diamond, Shaanan hops with the \D Q, I play the \D 4 and Shaanan's Queen holds the trick.

This is where things start to get weird. Shaanan starts playing all her heart winners and the rest of us discard. I carelessly discard my \D 9 when I should know that it's the only suit she can have remaining. On the last two tricks, she plays the \D J, Kelly throwing the \D A. Shaanan then plays the \D 7, Kelly playing the \C A with no one following suit.

``Oh no!!'' I say at the end. ``I threw away my \D 9 and let Shaanan make the contract and a beer!''.\\
Andrew: ``But she didn't win the last trick with it?''\\ 
Me: ``What do you mean?''\\
Andrew: ``Kelly played the \C A''\\
Me: ``But Shaanan had the lead?''

It seems that everyone at the table except me thought that Kelly won the second last trick with the \D A and then won the last trick with the \C A. I was so confused at this point because I did not think this at all. To me, it looked like Kelly through away the wrong card on the second last trick and Shaanan was lucky to make a beer at the end when both Andrew and I could have stopped her. 

They seemed to think that Shaanan played a diamond at trick 9 which Kelly won with the \D A. I tried to disprove that this is possible by counting out Shaanan's hand. She has shown up with two spades (\S AQ) and 6 hearts since Andrew and I had 3 while Kelly only had 1. Therefore she can only have two diamonds (\D Q7 specifically). I'm convinced what happened is that she played \D J at trick 9 and everyone thought it was a diamond except me (Shaanan did end up forgetting about the left bower on a later hand). The only possible way I think I could have got the layout wrong is if Kelly actually had 2 trumps or Andrew 4, but thinking it over I'm quite sure this wasn't the case. I think the layout looked something like this:

\setdefaults{err=off}
\gamefont{\larger}
\boardnr{1}
\northhand{9xx}{YBx}{Kx}{Kx}
\easthand{AQ}{bAKxxx}{Q7}{}
\southhand{T5}{QT8}{94}{987}
\westhand{KJx}{x}{Axx}{AJx}
\leftupper{}%
{}{}
\rightupper{\contract:
7\He}{\declarer: \east}{}
\rightlower{}{}{}
\showAll*
Anyway, I think I owe Shaanan a well-swindled beer!




\newpage

Playing at TSC with the OG crew (29/12/20). Usual pairings with Riley on my right.

\section*{Difficult Counting}

In the middle of a tight match, Bazli and I are down 2 games to 3. The previous game we sat at over 400 points most of the time and grinded until we were able to get over the top. This game was a tougher grind with both sides sub -400 and over +400 multiple times, usually at the same time. The previous hand the scores were -400 to -480 after I trapped the opponents in a 7$\Di$ contract where I had loads of defence. Riley picked up a 7-card club suit (with his partner having 3 card support!) so the scores are now -400 to -220. This is one opportunity we had to end the game.

Trying something new, we will look at the hand from Bazli's point of view. Sitting North in first seat, you pick up something along the lines of:
\begin{center}
    \par\noindent
    \hand{AJ}{T5}{AT9}{KJT}
\end{center}
Our goal when the score is this low both ways is to make safe bids, while still making the auction difficult for the opponents in the hope that we can give them a minus score. The auction goes:
\begin{center}
    \begin{bidding}
           & 6C & 6D & 6H \\
        7D & p & p & 7H \\
        8D & P\\
    \end{bidding}
\end{center}
Passing over $7\Di$ is quite disciplined which is good at these scores. You know that defending is going to go okay for you with the \S A, \C KJ10 and 3 diamonds to the Ace. That looks like probably two tricks with good potential for a third.

After picking up the kitty, John lays down the Joker and plays a diamond to Riley's \D J, Josh dropping the \H J. He says ``feels bad" as he loses his bower, but this is actually okay for our side since you have the \D A. Riley finishes drawing trump by playing a low diamond to John's \D K and your \D A (Josh pitches a spade). Double dummy this looks like a mistake since you have the highest outstanding trump, but Riley can't know this and John could very well have the \D A, so playing a trump here is a safe move (switch the \D A and the \D K, for example).

You're on lead now with not a lot of information early in the hand. What do you know about declarer's tricks vs your own and what should you return?

At the moment all we know is that they have an 8-card diamond fit, almost certainly 4-4 since Riley opened $6\Di$ and John competed to $8\Di$ (although with these guys you never know). That looks like 3 trump tricks and a ruff in either hand gives 4. On the other hand our side doesn't look likely to have many cashing tricks either, so an aggressive defence can only make life easier for declarer. So I think a passive \H 10 return is the safest option and this is what was played at the table.

Bazli plays the \H 10 to the \H K, \H 4 and \H 6. Riley plays \C A to the \S 6, \C Q and \C 10. Riley continues with a low club, John ruffing and Josh pitching a heart. John cashes the \H A with Riley pitching a low spade and John plays a low spade, leaving you with: 
\begin{center}
    \par\noindent
    \hand{AJ}{}{}{K}
\end{center}
What do you play now?

At the moment Riley has 7 tricks - 3 trumps, a ruff in John's hand, two hearts and a club. So we cannot give him another trick. There is only one more trump out and it is in Riley's hand. What can his distribution be? Riley is known to have 4 diamonds and a stiff heart. He almost certainly has a long club suit from his play, why else would he be playing a suit missing so many honours? So I think his two possible distributions are 1=1=4=4 (i.e 1 spade, 1 heart, 4 diamonds and 4 clubs) or 2=1=4=3. 

If Riley is a 2=1=4=3, then you need to hop with the \S A here in case he started with a doubleton spade honour and cash the \C K, which you can be confident will be the setting trick. The danger of hopping Ace is when Riley is 1=1=4=4 and this is what happened in the game. Bazli hopped Ace, Riley ruffed and exited a club to Bazli's \C K. Now Bazli was endplayed into giving up the contract with a spade to John's \S K! If he had ducked instead, Riley would still have to ruff (else give me the spade trick) and Bazli can retain his spade (and club) guard. 

I think counting out the shape of the hand is a reasonable thing to be able to do at the table, at least narrowing it down to these two options (I was doing it from my seat, at least). However knowing which case to play for is quite difficult. If you expect me (South) to have the \S K or \S Q (and John with the \S K) then ducking cannot lose. However I do think it is quite risky if you think Riley started with a doubleton spade. One place that might offer clues is the club suit. We know Riley has at most 4, Bazli had 3, John 1 and Josh a void. That's at most 8 accounted for, so there are at least two in the kitty. We also know that John kept a club when he maybe could have short suited himself, why would he do that? One possibility is that there are already three clubs in the kitty (giving Riley the 2=1=4=3). To me, that decision with the kitty doesn't make much sense when you could keep a long club suit. Why hasn't John short suited himself in spades? He must have something in spades, most likely the \S K? It's tough to think about what declarer put in the kitty, but I think there is some possibility for good inferences to be made. It's also very possible that I'm misremembering parts of this hand...



\setdefaults{err=off}
\gamefont{\larger}
\boardnr{1}
\northhand{AJ}{T5}{AT9}{KJT}
\easthand{T}{K}{BQ54}{A985}
\southhand{Q765}{Q984}{b7}{}
\westhand{K8}{A76}{YK86}{Q}
\leftupper{}%
{}{}
\rightupper{\contract:
$8\Di$W}{}{}
\rightlower{}{}{}
\showAll*


\subsection*{Uppercut!}

On the same night, the scores are 360 to 250 - advantage us. You pick up:
\begin{center}
    \par\noindent
    \hand{97}{Q654}{Q}{A95}
\end{center}
The bidding goes:
\begin{center}
    \begin{bidding}
           &  & 6H & p \\
        7C & 7D & 7H & P \\
    \end{bidding}
\end{center}
I have no interest in bidding on this hand at this score, especially after Riley opens $6\He$ on my right. I'm happy to let them play in hearts and Bazli seems to agree by letting them play at the 7 level. This is not dangerous for us and it looks like we have good defensive chances.

Before Riley puts away the kitty, he asks John what suit he bid. John says Clubs and that makes me feel good about my club holding scoring a trick. Riley leads the \H J, I play the \H 4 (count signal!) and to Riley's dismay, John pitches a low diamond while Bazli wins the Joker. That's a bad break! 

So far so good, but this is where it gets scary. Bazli plays the \S 8 (I'm unsure of what his actual pips were, but I know he played the \S 8 here). Riley thinks for a bit and then pitches a club. This is a generally good move on his part, giving his partner a golden opportunity to help out if he has anything in spades. I cover with the \S 9 and John wins the \S 10. He continues with the \S A (Riley pitching another club) and the \D A, dropping my stiff queen.

Counting declarer's tricks we know that he has 3 off-suit winners so far and at most 5 trump tricks (assuming he only has 6 trumps not 7, otherwise the contract is most likely cold). We need to hope that our \C A is holding up (likely after the comments before the kitty) and that we can score two more trump tricks. 

John continues with a club the Riley's \C K and my \C A. That's good! Playing a heart back would be suicide so I play back a club, fully expecting it to be ruffed and hoping to end play Riley for two trump tricks. He ruffs the club and leads the \D J, Bazli following low. He now thinks for a while and comes out a small heart. 

The thought here and the low heart exit automatically places Bazli with the \H A. This means Riley has the \H K and he has done very well to guess to lead low and force out the \H A rather than pin the \H Q by leading the \H K. Bazli wins the \H A and it looks like Riley is home if he can get the lead to draw trump. However, Bazli plays the \D K and I exclaim ``Oh my god trump promo! You legend Bazli!''. The final position is: 
\setdefaults{err=off}
\gamefont{\larger}
\boardnr{1}
\northhand{5}{}{K5}{}
\easthand{}{KT9}{}{}
\southhand{}{Q}{}{95}
\westhand{J}{}{}{QJ}
\leftupper{}%
{}{}
\rightupper{}{}{}
\rightlower{}{}{}
\showAll*

If Riley chooses to ruff low, I can over ruff. If he ruffs high, I can pitch a club and my \H Q is now promoted. This uppercut earns us the setting trick. Both of us congratulate each other for a very nicely played defence. It's often not easy to beat a 7-bid.

(On the actual hand, a spade lead in the 3-card ending is just as good as a diamond. The only way to lose is to play a club and I think Bazli didn't have any left, so it's possible he couldn't go wrong here).

\setdefaults{err=off}
\gamefont{\larger}
\boardnr{1}
\northhand{K85}{YA7}{K54}{6}
\easthand{}{BbKT98}{6}{KT8}
\southhand{97}{Q654}{Q}{A95}
\westhand{AJT}{}{A96}{QJ7}
\leftupper{}%
{}{}
\rightupper{\contract:
$7\He$E}{}{}
\rightlower{}{}{}
\showAll*







\newpage
\subsection*{Draw Trumps}

24/1/21 at the WSC with the OG Crew. Not a brilliant hand but a nice exercise in counting and defending.

The score is 40 to 380 their way and we pick up:
\begin{center}
    \hand{T876}{J94}{T87}{}
\end{center}
The bidding starts:
\begin{center}
    \begin{bidding}
        6D & 6H & 7D & ?\\
    \end{bidding}
\end{center}
What would you bid?

Even though our hand looks appalling, I think a {7\H} bid is clear under normal circumstances, and absolutely mandatory at these scores. If we get to steal the contract at {7\H} then that's a win for us no matter how many tricks we take. And if we make it, that's just a bonus. The seven level is the easiest place for us to steal points. Further, if they bid 8\D then I'm very happy to defend with my four trumps to the bower and club void. It's the perfect bid to put the opponents in an awkward situation.
\begin{center}
    \begin{bidding}
        6D & 6H & 7D & 7H\\
        p & p & 8D & P\\
    \end{bidding}
\end{center}
The bidding continues with a very slow pass from John (West), a slowish 8\D bid from Riley and a swift pass from me. Before picking up the kitty, Riley comments that his bid was designed to push us up. So was mine, I say.

Riley begins by leading the Joker, I play low (count - if anyone is paying attention) and Bazli drops the \D Q. Riley continues with a low diamond, I play small and hold my breath (trying to act cool) as John wins the \D J and Bazli discards the \S 9. John takes a second to think and leads the \S A to the Jack, Queen and Ten (notice the 9 has already gone so dropping the Ten here is safe and shows our sequence). John continues with the \S K, Bazli and Riley both pitch a low heart and I drop the \S 8 (although the proper card is the \S 6, showing the rest of the sequence). 



Questions for the defence (you):

1. What is the trump break?\\ 
2. What was the original spade layout?\\
3. What is Riley's (declarer's) distribution?\\

1. Almost certainly 4-4 from the bidding and play so far. It's unlikely that John has the \D A since he might have finessed (or at least thought about it, playing to drop the \H J is an option too).

2. You can count the whole suit to be \S J9 with Bazli (North), \S Q with Riley (East), leaving \S AK(5) with John (the \S 5 could be in the kitty). 

3. If Riley's heart discard was a singleton, then he must be 1=1=4=4. Although it is possible he has two hearts, making him 1=2=4=3. 

John continues with a heart to the \H Q, \D K and \H 9 (confirming Riley is 1=1=4=4), leaving you with 
\begin{center}
    \hand{76}{4}{bT}{}.
\end{center}
What is your plan for the rest of the hand? How are you going to defeat this contract?

At the table, my plan was to get in with a club ruff, draw Riley's last trump and cash a heart trick (Bazli's \H A) to beat the contract. This almost happened. Riley lead a club, I ruffed in, cashed the \H J and then lead a small heart. To my surprise, John ruffed the heart. I thought oh this is fine, he must still have a spade, showed him my last two cards (\S 76), and he showed me the \C K and the \S 5 (upset that my \S6 was higher than his \S5 haha). Bazli had \C Ax and Riley \C QJ, so we agreed on one off. 

It looks like I got lucky here when I guessed John's last major card wrong that he couldn't cross to Riley's hand and cash two clubs. If I guess whether he still has a spade or a heart, then we instantly have them 1 off. Although, if John has the \C A, he would still be end played to give us a spade or club trick (unless he unblocks it!!). In any case, it's not very likely that Riley has underled the \C A (although maybe he should to keep communication) and John is probably stuck in his own hand with a loser even if Bazli doesn't come through for me with the \C A.  

\setdefaults{err=off}
\gamefont{\larger}
\boardnr{1}
\northhand{J9}{AKQx}{Q}{Axx}
\easthand{Q}{x}{YAKx}{QJxx}
\southhand{T876}{J94}{T87}{}
\westhand{AK5}{x}{Jxxx}{Kx}
\leftupper{}%
{}{}
\rightupper{\contract:
$8\Di$E}{}{}
\rightlower{}{}{}
\showAll*

Looking back on the hand, Riley/John could have played it much better following basic principles. At trick two (after cashing the Joker), Riley should lead the \D A instead of a low Diamond, especially after seeing the \D Q drop. This tells John that Riley has the \D K and invites John to take a finesse through me. It's important to guess right here, because if the finesse loses, you will lose a heart and a club soon after. I think the percentage play is to finesse since Bazli could be false carding. If John had elected to take this finesse, then it looks like they will make the contract easily as long as they dispose of their heart loser quickly and then play on Clubs. They will only lose one Diamond and one Club, so long as the play Clubs early.


Cashing the top two spades quickly is a nice play here from John because it allows Riley to get rid of his heart loser quickly (although it would have been nice to not crash the honours). The only problem is it has set up my spade winners. This is fine as long as Riley can ruff, although John made a mistake by making Riley ruff a Heart too early. This cause them to lose control of the hand before they set up the Club suit. Playing double dummy, you can avoid this problem by not crashing the Spade honours, although who is going to play a Spade before drawing trumps?



\subsection*{Making Defenders Guess}

You're playing {8\C} with this hand (post kitty):

\begin{center}
    \hand{K7}{A9}{}{YBKT76}
\end{center}

The auction was 

\begin{center}
    \begin{bidding}
          & & & 6C \\
        6H & p & p & 7C\\
        7H & p & p & 8C\\
        P\\
    \end{bidding}
\end{center}

What do you play at trick one?

At my table, John (LHO) made some sort of comment that I had bid too high. Taking this on board, I imagined that he might have the \S J and a number of clubs. To try to get him to duck the first trick, I led the \C 10, hoping Bazli had the \C K or \C Q and I could get John to duck the first trick. I think this is the correct play irrespective of whether anything is known about the suit.

John discarded on the first trick, so turns out he was just joking around. Bazli won the \S J (bonus!) and played a club back. Riley put in the \C Q, I won with the \C K and played the Joker, revealing that trumps are 3-3. We are now up to seven tricks - 6 trump tricks and the \H A. How do we maximise our chances of an 8th trick?

I don't want to play a spade because I could be killing my chances in the suit. I want spades led towards me to guarantee a trick in the suit. If Bazli has the \S A or \S Q, we will always get a trick, and if he has both then not playing Spades only costs an over-trick. So I'm definitely playing Hearts. I know lots of players that will cash their \H A and think about the rest later, but this makes it too easy for the opponents and makes you very easy to read. Instead, I lead a low heart in case I could get John to duck the \H K and Bazli to win the \H Q.

 This is exactly what happened at the table. When I lead the low Heart, John stared at \H KJxx and had to decide what to play. Playing low looks like giving up on the suit (no one else is likely to have long Hearts), playing the \H J is correct when I have the \H Q and playing the \H K is correct when I have the \H A. He played the \H J and Bazli won the Queen! I don't blame him for his choice because not a lot of players underlead Aces. I've essentially given the defence a King-Jack guess (like in bridge) and he guessed wrong. Maybe next time he will consider going up with the \H K.

 The best thing about this hand is that Riley showed up with the \S AQ, so I was not going to win a Spade trick if I had to play them myself (although who knows, John still may have played a Spade back if he had guessed the Heart suit correctly). I say this because if the \S A was to my left, some players will fly Ace if I lead a Spade. So on good defence, the extra little play might have been the difference between making the contract and going out the bog.



\newpage
\subsection*{Double Finesse}

A Sunday night (18th July 2021) at the WSC, playing with Shaanan across from me, John on my right and Bazli on my left. John bids Open Misere (something like his fourth one already) and you hold:
\begin{center}
    \hand{T876}{K}{AK5}{A6}
\end{center}
The lead is the \C 7, you oblige with the \C 6 and partner wins the \C K. Dummy comes down, and you are quite impressed with John's hand for once.
\setdefaults{err=off}
\gamefont{\larger}
\boardnr{1}
\northhand{}{}{}{}
\easthand{}{}{J964}{QJ9875}
\southhand{T876}{K}{AK5}{A6}
\westhand{}{}{}{}
\leftupper{}%
{}{}
\rightupper{}{}{\contract: OM}
\rightlower{}{}{}
\showAll*

Partner plays the \D 7. How do you plan to beat this contract?

It doesn't look like you have much chance, but don't give up hope! Dummy plays the \D 6 and you win the \D K. Partner is going to need to have the \D 8 in order to knock out the \D 4. But if you lead the \D 5 now, the \D 9 will be lowest outstanding. So you will need the diamond lead to come through dummy.

How do you get to partner's hand? Unfortunately, we don't have a low club any more, so it'll have to be a low spade. Don't play a heart, even if they have the \H A partner might not know to overtake! It looks scary that dummy will get to pitch on diamond on the spade, but this is safe since you play to get the on the third round, not the fourth.

You play the \S 6 to the \S A and the \D J. Partner comes through for you with the \D 8! John plays the \D 4 and you win the \D A. Now what?

If partner's original holding was \D 87 doubleton, you can win now by coming out with the \D 5. Can we do better than that? At the table while trying to figure out this hand, I envisaged partner with \D T87 because I thought at the time that the \D T may be a relevant card. With this in my, I cashed the \C A first just in case parter started with a stiff club and could discard their \D T. This proved necessary today. Shaanan threw away her \D T (without hesitation!), I played the \D 5 and she showed out while John won the \D 9. A beautiful defence executed with perfect timing.

Notice that timing our entries properly is essential to this hand. In principle, we always want North (right of dummy) to win the opening lead so they can play through the dummy. Hence playing low at trick 1 was a vital play from me. Shaanan did really well to attack John's diamond weakness right away, realising that it will be important to lead from her hand while she has the chance. She also did well to play the \D 7, not the \D T. Notice that I cashed the club as late as possible in the hand so that partner knew exactly what to discard. I could have cashed it earlier, but it would not have been obvious that partner needs to unblock the diamond. After taking two finesses, the need for this becomes obvious.

On the second finesse when Shaanan plays the \D 8 through John's \D 94, John should stop and consider his options (with the \D 7, \D 6 and \D K already played). Playing the \D 9 loses whenever I have the \D 5, but wins whenever Shaanan has the \D 5. Although, if Shaanan has the \D 5 then playing the \D 4 still might win because I may not have an entry to her hand (we would need some strict club layout or for her to have the Joker). It also picks up cases where I have the \D 5, but cannot short suit Shaanan. So I think playing the \D 4 and making the defence prove their worth is the correct play.


\subsection*{We Need a Squeeze!}

After changing partnerships, I'm playing with John across from me and Shaanan on my right. The scores are roughly 300 to 0, and you pick up this monster.
\begin{center}
    \hand{AJT75}{}{A4}{KQJ}
\end{center}
I can see seven tricks in spades with no help from partner or the kitty, so making at least 8 tricks seems pretty likely and an easy way to get the 200 points we need. So I decide that I am going to bid 8S as my first bid. Unfortunately, Shaanan gets to bid before me.
\begin{center}
    \begin{bidding}
          & & 6S & 8S \\
        OM & TN & P\\
    \end{bidding}
\end{center}
Well that escalated quickly.

I apologised to Shaanan and assured her that I wasn't messing with her, I actually wanted 8S. We had a few discussions about leading and discarding \textit{top of a sequence}, but John still goes ahead and leads the \H J. Yes he is a mad man, but no he is not doing this without the \H AKQJ at least. As John plays off two more rounds of Hearts, Shaanan follows to two rounds only and Bazli follows to the third round with the \H 10. What are your prospects for making this ambitious contract?

In 500, there are 13 cards in a red trump suit. Minus 2 for the Joker and left bower leaves 11 cards in a red suit at no trumps. The opponents have shown up with (only) five of them, so John has six heart tricks in his hand. Those six tricks plus his Joker and my two Aces gets us to nine tricks, so we need one more. If partner has either of the pointed kings or the \C A the hand is going to be a breeze. Without those cards, are there any other chances?

From the auction, I imagined Shaanan with both the \S K and \S Q along with some length. Since we had only nine tricks, I hoped that she had the \C A so I could play for a squeeze. Some layout like this is what I was imagining:
\setdefaults{err=off}
\gamefont{\larger}
\boardnr{1}
\northhand{x}{YAKQJ97}{x}{x}
\easthand{KQxx}{xx}{xx}{Ax}
\southhand{AJT75}{}{Ax}{KQJ}
\westhand{}{}{}{}
\leftupper{}%
{}{}
\rightupper{}{}{}
\rightlower{}{}{}
\showAll*
Cashing out our tricks in the right order (using the Joker for an entry back to John's hand) would give:
\setdefaults{err=off}
\gamefont{\larger}
\boardnr{1}
\northhand{x}{7}{}{x}
\easthand{KQ}{}{}{A}
\southhand{AJ}{}{}{Q}
\westhand{}{}{}{}
\leftupper{}%
{}{}
\rightupper{}{}{}
\rightlower{}{}{}
\showAll*
The lead of the \H 7 squeezes Shaanan in Spades and Clubs. It's great that I hold both of the threats, so I exactly what to discard and can play the squeeze without help from John. Although, it would be nearly impossible for me to convince John to cash the diamonds before squeezing me, and to not break out communications in spades. But I can dream that I'm playing both the hands. I think there are a few interesting squeeze options in positions like this that can abuse the power of the Joker, but I haven't exactly worked out the details.

Having seen this idea early, I decided to discard the \C K to tell partner about my honour cards and hopefully he could figure out what to do. John then played a club, and we had a discussion about what our discards should mean. Unfortunately, I didn't get the layout I wanted anyway.

\setdefaults{err=off}
\gamefont{\larger}
\boardnr{1}
\northhand{}{YAKQJ97}{Q4}{6}
\easthand{KQxx}{}{86}{QJ9875}
\southhand{AJT75}{}{Ax}{KQJ}
\westhand{Q}{T54}{Kxx}{Axx}
\leftupper{}%
{}{}
\rightupper{}{}{\contract: OM}
\rightlower{}{}{}
\showAll*

Without spade communication and the diamond threat offside, the squeeze opportunities look pretty hopeless.


\newpage
\subsection*{More Than Anything}

Shaanan and I have just moved to Adelaide and we're playing with Bazli and Riley on Trickster (6th of Feb 2022). Shaanan and I are down about 0 to 300 and I pick up:
\begin{center}
    \hand{x}{x}{AQJT9864}{x}
\end{center}
The bidding went:
\begin{center}
    \begin{bidding}
         & & 6H & 8D\\
        P \\
    \end{bidding}
\end{center}
The Eight Diamonds bid is quite rash, but I think the bridge player in me thought I had a better suit than I actually do. It is good to get in the opponent's face and put the pressure on though. I could tell Riley had a problem because he took a long time to pass. I get the \S K, a Heart and a Club in the kitty so my hand becomes.
\begin{center}
    \hand{Kx}{}{AQJT9864}{x}
\end{center}
I lead out the \D J in case Riley has a stiff bower, Bazli drops the \D K while Shaanan follows low and Riley wins the Joker. Riley tries the \H A which I ruff and lead out the \D A. Bazli discards a high spade, Shaanan follows low and alas Riley wins the \H J. Riley goes into the tank and eventually comes out a low Spade.
\begin{center}
    \hand{Kx}{}{QT984}{x}
\end{center}
We've already lost two tricks - how are you going to make this hand?

I decided to play Shaanan for AQx in Spades. Having dodged the bullet of Riley returning a club, I flew with the \S K to draw trumps and keep a late entry into partner's hand. Thankfully, the \S K held and I knew I was in business. I drew the last trump from Riley, happily noting that Shaanan did not pitch a spade. In order to avoid giving her a discarding issue, I played a Spade to her Ace. After pausing for a few seconds, the moment of truth arrived when Shaanan banged down the \S Q. I blurted out: ``Shaanan I love you more than anything!'' and threw away my Club loser as Bazli threw away his \C A while complaining to Riley that he had discouraged Spades. Hands like these highlight the importance of defensive communication and signalling.

Even if partner held only Axx in Spades, we still have a good chance at three Spade tricks if the opponents mess up the discarding. With a counting partner, I could run the trumps to try and squeeze Bazli (or Riley) if he holds Spade length and the \C A. It would be legitimate over Bazli if Shaanan held the \C K, but I think it would work no matter what Club holding Shaanan has because neither of them is ditching the \C A. I think I would have to be partnered with a bridge player in order to trust them to hold onto their Spade threat for me.




\newpage
\subsection*{Never Give Up!}

Playing on Trickster Compete with Shaanan (23rd April 2022) in the little boy leagues. Shaanan has been super into playing with me on Trickster recently, which has been fun although we are constantly getting accused of cheating. I think this is mostly because we play together often, although it is pretty funny when I get accused of cheating when I make really good plays. You pick up something along the lines of:
\begin{center}
    \hand{BAQ85}{}{Kxxx}{x}
\end{center}
This is your hand post kitty, at least (I can't remember what it was pre-kitty). The auction goes something along the lines of
\begin{center}
    \begin{bidding}
         & 6N* & 7H & 8S\\
        P\\
    \end{bidding}
\end{center}
We've agreed that 6NT shows Joker and an Ace or three Aces. Assuming partner has the joker, I thought my hand would play pretty well, so I jumped to 8 Spades. 

You lead the \S 8 (reverse count), and leftie flies with the Joker while Shaanan shows out! Welp, so much for draw trumps and set up Diamonds... Leftie plays a small Diamond and Shaanan flies up with the \D A (seems like nobody's ever heard of second hand low). I thought this Diamond lead was a singleton and was starting to lose all hope for the contract. Shaanan goes into winner cashing mode, \C A, \C K on which East pitches a Heart and I throw a Diamond, then the \H A on which I throw my last little Diamond. After a brief think, Shaanan continues with a low diamond, on which East pitches another Heart, I win the \D K and West follows! Down to \S BAQ5 and one loser left, what do you play?

Without too much consideration I came out the \S 5! When the \C J popped on my left I announced to Shaanan "Holy shit I'm making on a trump coup!". Sure enough I got to overruff leftie's trump with my \S Q and claim 8 tricks.

On reflection, playing a low Spade here is an awful play. Playing the \S J and then playing a low spade is much better because it picks up \C J doubleton either way or 4-3 trumps with no trump promotion available. I think my pessimistic outlook on the hand stopped me from finding this play. 

There were lots of mistakes other than mine on this hand. Leftie should clearly stick in the \C J on the first trump, but people love to play the Joker as soon as they can. I quite like Shaanan's play of flying \D A since she has so many winners to cash, although it is a spot to consider ducking. Rightie should figure out that it is safe to ruff the second diamond, even though it is unnatural for him to want to ruff from his 5 card trump suit. I have shown out in hearts and clubs, and if I have only trumps left there is nothing he can do anyway. With a little counting, this should be a straightforward play to find. I will have no hope from here, since I'm getting endplayed in trumps.


\newpage
\subsection*{Watch the Discards}

Playing on Trickster with Bazli (25th April 2022) in the big boy leagues, first hand of the set I pick up.
\begin{center}
    \hand{xxx}{AQx}{JT}{Kx}
\end{center}
The bidding goes:
\begin{center}
    \begin{bidding}
         & & 6D & 6H\\
        8C & 8D & p & 8H\\
        9C & OM & p & p\\
        10C & P \\
    \end{bidding}
\end{center}
Since this is our first match against these people, I don't know if the bid Aces and whether or not 8D is a cue-raise or natural. I think playing with randoms we should assume cue-raises are off because these people can bid on anything. That being said, I nervously bid 8H. The fun did not stop there, and we ended in 10C. 

Declarer starts with the top two trumps, Bazli discarding a high diamond. Noting this, I'm thinking about whether I will need to guard diamonds in the end position. Next, declarer cashes the \S A and continues with the trump barrage. Bazli discards a lot of diamonds, although with things moving so quickly its hard to keep track of them exactly. In the three card ending, you hold \H A \D JT. On the last trump, Bazli pitches the \D Q, rightie throws the \H J and you discard the ... ?

After a long (29 second) think I chose to discard the \H A. I thought declarer probably has the \D A for his determined bidding and would have kept a small diamond with it from the kitty. Sadly, leftie led a low diamond to his partner's \D A who then cashed his \H K for the 10th trick. On this, declarer threw the \S Q and Bazli the \S K. 

Maybe if I had kept track of all the discards I could have figured out the end position? I'm not so sure since the two missing Diamonds could have easily been with declarer. What about the Spade position? Partner would have discarded the \S K at some point if he held both missing honours to help clarify the position, although I was assuming that declarer had no more Spades, otherwise why is he playing the \S A so early? I think I subconsciously decided that Spade was a singleton. I always decided that East was holding onto the \H K because of his late \H J discard.

The clue that solves the problem for me is that East opened 6D and parter made a certainly natural 8D bid (since he discard so many Diamonds and didn't have any Heart support). If I watch the discards more closely, I can probably conclude that East bids Aces and was showing the \D A. From there, it becomes and easier game of trying to keep what he keeps. Even if he doesn't have the \H K, I doubt the opponents are capable of coordinate a legitimate squeeze. To his credit, leftie did put me under a lot of pressure. 

The three card ending looked something like this:
\setdefaults{err=off}
\gamefont{\larger}
\boardnr{1}
\northhand{Kx}{}{Q}{}
\easthand{}{KJ}{A}{}
\southhand{}{A}{JT}{}
\westhand{Q}{}{x}{5}
\leftupper{}%
{}{}
\rightupper{}{}{}
\rightlower{}{}{}
\showAll*
If East had kept two Diamonds instead of two Hearts, then this is still an illegitimate double squeeze because the heart and spade threats are the wrong way around. 

\subsection*{Suspicious}

Still playing with Bazli on Trickster Compete, post-kitty you hold 
\begin{center}
    \hand{}{}{ATx}{bAKxxx}
\end{center}


\newpage

\subsection*{What Do You Call a Nine Card Suit?}

Playing at Guey's house (4/11/2022) partnered with Shaanan, Gue on my right and Bertie on my left. I've recently introduced these two to 500 and we've been having a great night. This game has been dragging on for a while though and Shaanan has hinted a couple times that she's getting tired and ready to go home. Scores are 140 to -30 our way. It was 460-460 a couple hands ago, but we went off in a semi-block 8 Diamonds and Guey went off in a misere and an open misere consecutively.

I'm just a bystander today with an irrelevant hand witnessing the auction escalate quickly. Shaanan opens misere confidently in first chair, Guey bids 8 Clubs(?) and Shaanan competes to open misere. Guey thinks briefly about bidding ten but decides to pass. I ask Shaanan if I can take a peek at her hand and walk round the table to see this
\begin{center}
    \hand{}{AKQJT9654}{6}{}
\end{center}

``HOLLYYY...''

She put the kitty straight back and led the \D 6. I nervously checked my Diamond holding and was disappointed to find KT tight. Guess we're praying for blocked diamonds! 

Guey thinks for a bit and wins the \D A, holding A54. Boom! He plays a Club, Bertie Jokers it and smashes down the \H 7. I explain to them that Shaanan will just play the \H 4 and you can't beat the contract. 

It turns out that the beer card was in the kitty and we joked that Shaanan should have kept that instead of the \D 6. This would have been funny but is actually a horrible blunder. If Guey only has one of the 4 or 5 of Diamonds, you want to convince him to hold onto it. Leading the 7 makes him more likely to play the 4 or 5 on face value. On the lead of the 6 it could be right to hold up if Shaanan has long Diamonds. After all, she has done very well to conceal her nine card suit. 

You would have to be playing a very deep game to lead the \D 7 in this spot. 
















\end{document}
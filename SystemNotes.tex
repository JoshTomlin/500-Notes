\documentclass[a4paper]{JoshCards}

\title{System Notes for 500}
\author{Josh Tomlin}
\pagestyle{fancy}
\date{}

\begin{document}

\maketitle

\chapter*{Implemented Conventions}

\section*{Open Misere Force}

After partner has bid misere, Open Misere Force is the agreement that any nine bid is a cue bid showing a misere control, forcing to Open Misere. A Misere control in a suit is one of the two lowest cards (i.e a 4 or 5 in a red suit or a 5 or 6 in a black suit). After you bid 9X, partner will relay by bidding one step up, giving you the option to show another control or transfer to Open Misere by bidding 9NT. This is the most beneficial 500 conventions we have come up with and we use it all the time.

When forcing partner into Open Misere, you should have controls in at least two suits (otherwise just pass, unless you're desperate). However, the simple `bid your lowest control' system does not allow you to show two touching controls. Thus I have modified the system so that when you have (exactly) two touching controls, you bid the higher one then sign off immediately. Assuming you would only force with at least two controls, this implies a control in the suit directly below the one you bid. If you show two controls, then you are showing exactly two controls and denying controls in any other suits.

\begin{table}[h!]
  \begin{center}
    \label{tab:table1}
    \begin{tabular}{|l|c|l|}
      \hline
      \textbf{Response to Misere} & \textbf{Misere Bidder} & \textbf{Forcer Rebid}\\
      \hline
      Any 8 bid: Natural, to play & Pass &\\
      &&\\
      9\S: \S & (9\C): Relay 
        & 9\D: \S\D, no \C\H\\
        && 9\H: \S\H, no \C\D\\
        && 9NT: \S\C, and $\D$or \H\\
      &&\\
      9\C: \C, no \D & (9\D): Relay 
        & 9\H: \C\H, no \S\D\\
        && 9NT: \S\C, no \D\H\\
      &&\\
      9\D: \C\D, no \S\H & (OM): Sign off & \\ 
      &&\\
      9\H: \D\H, no \S\C & (OM): Sign off & \\ 
      &&\\
      
      9NT: \D\H, and $\S$or \C & (OM): Sign off & \\
      \hline
    \end{tabular}
  \end{center}
  \caption{\textit{The Open Misere Force Structure.} Bids in brackets are made by the misere bidder, all other bids are by the forcer. The goal is to end in Open Misere, so once you have the required information you just bid Open Misere. The rule to remember is: if you have two touching suits, bid the higher one then sign off (i.e bid 9NT). To show three suits, either bid 9NT or 9S followed by 9NT to show both reds and one black or both blacks and one red respectively. Systemically, showing only two suits denies controls in the other suits, but you can opt to take this approach with three suits if you think it is necessary. With four controls, choose which way you would like to show three of them.}
\end{table}

There are only two ways to show a hand with three controls, due to the stringent room restrictions of the 9-level. The first way is to bid 9S followed by 9NT and the second is to directly bid 9NT. I have decide that 9S-9NT should show a control in both black suits and one red control, while a direct 9NT bid shows two red suits controlled and one black suit. This ensures that 9S definitely shows a Spade control, in case the opponents interfere in the auction.


\newpage 

\section*{Cue Raises and Cue Bidding}

When the opponents have bid a suit, I do not recommend trying to also bid their suit naturally. This is especially true if you are playing with no misere restrictions, since they are less likely to be messing around. Therefore, we can use the opponents suit as an artificial cue bid. The most common use of this cue bid in bridge is to show partner strong support for their suit. This allows you to show partner that you have a good hand while keeping the auction at a low level, and allows us to start showing each other first round controls (Aces and voids). 

Systemically, a cue-raise shows partner four card trump support and a good hand that wants to play at atleast the 8-level (possibly thinking about going higher). A good hand can include features like strong trumps, lots of off-suit, a good source of tricks or shortnesses (singletons/voids). Cue-raises are forcing to the 8-level.

For example, suppose LHO opens 6S, partner bids 6H and RHO passes. With a strong hand you can make a cue-raise of 7S. You have now established what sut will be trumps and a force to the 8-level while keeping the auction below 7H. We now start cue-bidding first round controls (Aces/voids), including cue-bidding No trumps to show the joker. You always make the lowest available cue bid first, so if you skip a step you are denying a control in that suit. After LHO passes in our current example, 7C from partner shows a FRC in clubs, 7D would show a FRC in Diamonds while denying one in Clubs, 7NT shows the joker while denying Diamonds and Clubs and so on.

A bid of 7X (X = partner's suit) or misere during a cue-bidding auction is an optional waiting bid, saying ``I have nothing to cue below 7X/misere, I don't have much to say and I'm waiting to hear from you". This could be a useful option if you are looking for just one card from partner in order to make a big bid. The information of what Aces partner does and does not have is so valuable in the play, and can change how you evaluate your hand. It is especially useful when declarer has to make decisions about what to keep from the kitty.

If available, a jump cue in the opponents suit is a transfer to partner's suit. That is, jump cueing shows a good enough raise to play at the jump level, but you want partner to play it (usually because they have longer trumps and the kitty will be more useful to their hand). In the above example auction of (6S)-6H-(P), you can bid 8S as a transfer to 8H for partner. This should show a hand good enough for 8H, but with no interest in going higher. This also may be done on a strong 3-card suit, where partner might need any extra trumps found in the kitty.

If after a cue-raise you know that you want to play at least 9X and are thinking about 10X, they you can use RKCB (see Experimental System Notes). This would be a perfect spot for such a convention, including the use of kickback since the trump suit has been clearly set.


\newpage

\chapter*{Experimental Conventions}

\section*{Keycard}

\subsection*{Roman Keycard Blackwood (RKCB)}

There are six keycards in a suit contract - the three offsuit Aces (I will refer to them as OAces when convenient), the Joker and the two Bowers (as opposed to five keycards in bridge). After the partnership agrees on a trump suit, 8NT is a keycard ask. This is an artificial bid (alertable) and forcing. The responses are (1430 style):

{9\S}: 1 or 4 keycards\\
{9\C}: 0 or 3 keycards\\
{9\D}: 2 or 5 keycards, no Joker\\ 
{9\H}: 2 or 5 keycards, including Joker.

This convention is lets you figure out how many keycards your side holds when you are trying to bid a grand (a 10 bid). You must be ready to play at at least the 9 level, and if partner shows you all the missing keycards you should be bidding 10 (otherwise why are you asking). Here I include the left bower as a keycard (slightly different to bridge where the Q of trump is not considered a keycard) because it is necessary to know about when trying to bid a grand. There is no response to show all 6 keycards: this is because I can't imagine a situation where someone is asking for keycards without any of them. If this does happen to come up, responding 9NT should suffice.

In the same way you can ask for the Queen in bridge, you can ask for the Joker. If Responder bids {9\S} or {9\C}, Asker can bid one step up ({9\C} or {9\D} respectively) to ask Responder if they hold the Joker. Responder signs off in 9 trump if they lack the joker, or can bid one step up if they have it (skipping 9 trumps of course, since that would be a sign off). See the Specific Ace Ask subsection for what to do when you are missing exactly one keycard.


\subsection*{Kickback Variation}

When using standard RKCB, if the agreed trump suit is not hearts, then some of the 8NT responses are above 9 trump. This risks being pushed to 10 trump when some important keycards are missing. Kickback solves this problem by starting the keycard conversation at a lower level when the agreed trump suit is lower - saving space. 

After a trump suit has been agreed, bidding one step above 8 trump is kickback. For example, if Diamonds is agreed trumps, then {8\H} is kickback. If Clubs is trumps, then {8\D} is kickback and if Spades is trumps then {8\C} is kickback. When someone bids kickback, they are making a RCKB enquiry (as if they bid 8NT, which is the kickback suit when Hearts is trump). The responses are the same as RKCB, except they are encoded as steps above the kickback bid.

1 Step up: 1 or 4 keycards\\
2 Step up: 0 or 3 keycards\\
3 Step up: 2 or 5 keycards, no Joker\\ 
4 Step up: 2 or 5 keycards, including Joker.

Asker can also enquiry about the Joker in the same way as normal RKCB.
\todo[inline]{Situations where there is no agreed upon trump suit.}

Sometimes, there is no agreed up trump suit yet, but a bid can still obviously be kickback. The assumed suit is the most recent suit that responder has bid. However, if there is no agreed upon suit yet, then bidding a suit partner has already bid is natural \todo{Maybe??? might want kickback there though idk} and never kickback.
\todo[inline]{Need to fill everything with examples}


%Over a six bid without competition, there can never be any confusion between a %splinter and kickback. This is because a jump in another suit must always be below 8 %trump. However there could be confusion with splintering 'at the 9 level' (i.e a %splinter that forces 9 trump). This is a situation that might frequently occur in a %competitive auction. Possibly, we can use 8NT to represent a splinter in the %kickback suit in situations where such a 9-level splinter bid would be kickback. The %bid 8NT can also be used to replace a cue bid that is taken up by kickback. In %essence, it's as if we've switched the 8NT bid with the the bid one step above 8T. %This saves the necessary space when keycarding, whereas splinters and cue bids don't %necessarily need this space.


\subsection*{Specific Aces}

Suppose you bid keycard and find out that you're missing one keycard, but your side has the joker and maybe the Right bower. Since you start with the lead, there is still a good chance you can make 10 tricks with a good hand if you keep the right suits after the kitty and choose the right suits to play. For example, you might be missing an Ace in one suit, but have another running offsuit so that the missing Ace doesn't matter. Hence, I think it can still be quite valuable to find out which keycards partner has. The beauty of 500 is that the Open Misere bid gives us a little bit of extra room to find out this information efficiently.

For simplicity we assume that hearts has been flagged as the trump suit (i.e 9NT is the kickback suit), although this same structure can of course be used with Kickback by shifting every bid accordingly. We assume that Asker has found out that their side holds a total of 5 keycards including the Joker.

\newpage
\begin{table}[h!]
    \begin{center}
      \label{tab:table1}
      \begin{tabular}{|l|l|l|}
        \hline
        \textbf{Response to 9NT} & \textbf{Asker Rebid} & \textbf{Responder Rebid}\\
        \hline
        10\S: \S A $\mid$ no \C A \D A & (10\C): Inquiry  
        & 10\D: \S A $\mid$ no \C A \D A\\
        \qquad (OR: \S A \C A \D A)
        && 10\H: \S A \C A \D A\\
        &&\\
        & (10\D): \H A? Soft trumps & 10\H: \S A $\mid$ no \H A or not worried\\
        && 10NT: \S A \H A, worried\\
        &&\\
        & (10\H): Sign off, \D A&\\
        \hline
        10\C: \C A $\mid$ no \S A \D A& (10\D): \H A? Soft trumps
        & 10\H: \C A $\mid$ no \H A or not worried\\
        && 10NT: \C A \H A, worried\\
        &&\\
        & (10\H): Sign off&\\
        \hline
        10\D:  \D A $\mid$ no \S A\C A & (10\H): Sign off &\\
        &&\\
        10\H: no \S A \C A \D A && \\
        \hline
        OM: Exactly 2 OAces & (10\S): Relay, no OAces   
        & 10\C: \C A \D A $\mid$ no \S A\\
        && 10\D: \S A \D A $\mid$ no \C A\\
        && 10\H: \S A \C A $\mid$ no \D A\\
        &&\\
        & (10\C): \H A, Soft trumps & 10\D: not worried (transfer)\\
        && 10\H: worried (transfer)\\
        && 10NT: worried\\
        &&\\
        &(10\D): \H A? Confirms all OAces & 10\H: no \H A or not worried\\
        & \qquad \qquad \quad (worried about trumps) & 10NT: \H A, worried\\
        &(10\H): Sign off, one OAce&\\
        \hline
      \end{tabular}
    \end{center}
    \caption{\textit{The Specific Ace Ask structure.} Bids in brackets are made by the asker, all other bids are by responder. In theory, the partnership should hold exactly 5 out of 6 keycards, hence after a {10\S} response Asker should know which case it is. However, this might not be the case in practise, so I have included an inquiry {10\C} bid. Enquiring about the \H A should confirm all the off suit aces and show worry about the trump suit (most likely missing the Right Bower). Responder can figure out which bowers are missing from the suit, knowing that Asker cannot have the \H J. They should then do something intelligent if they have the \H A, deciding whether to stick it out in {10\H} (maybe with the \H J and/or some extra trump length) or try 10NT (maybe with some some suits that are running after hearing about partner's Aces). After bidding OM, you show the OAce you don't have with 1 step being low, 2 steps being middle and 3 steps being high. The way to remember this system is that the {10\S}, {10\C}, {10\D} shows exactly that Ace (or sometimes all three in the case of {10\S}) and if you have 2 Aces you go through OM. Also {10\D} is a \H A inquiry.}
  \end{table}

  In order for this structure to work when a suit other than hearts is trump, you need to be playing kickback. In that case, the bids are all shifted down by however many steps the kickback suit is below 9NT. For example, if Diamonds are trump, shift every bid down by 1. If clubs are trump shift down by 2 and if spades are trump shift down by 3 (OM is included as a normal bid in this shifting math). 

  Knowing how many keycards partner holds and what aces they hold should tell Asker about the trump suit by implication. Knowing what high trumps partner holds and how to get to their hand can tell Asker exactly how to play the trump suit (maybe they have to guess which bower they have). This means declarer at trick one can decide the best way to pick up the trump suit and knows how to get to partners hand for a first round finesse, say, if needed. 

  For example, suppose you have a 5 card trump suit headed by the joker and an outside Ace. Partner has shown support and shows you 3 keycards including two outside Aces using the Ace ask structure (this time I don't care what suit they are in). Thus your side is missing exactly one of the bowers. At trick one, you put partner on lead with one of their aces (leading a high card to indicate you want a switch). Partner will now know how to play the trump suit. They know you have the Joker since you forced grand but did not ask about the joker (they have to play you for the Joker anyway). If they have the Right Bower, they will lead it and lead towards you (hopefully giving you count so you can decide whether to play for the finesse or the drop). If they don't have the Right Bower, they should lead the Left Bower (hopefully one of you has the Ace of trumps) so you can finesse. So even if you bid a grand without a solid trump suit, having all this information helps immensely in getting the play of the hand correct.

  The other case where this will be useful is when your side is missing one offsuit Ace. Declarer finds out what Ace this is and keeps their high cards in the other suits. You now never worry about going off by randomly playing the wrong suit, and have a lot more possibilities to play for in the other suits (say, taking a finesse or trying to cross ruff on the weak suit). This will at least remove the thought of playing for layouts where your partner is missing the necessary Ace, and points you in the direction of finding other layouts to play for.



  An Open Misere ask (instead of a 9NT ask) could be used as some sort of asking for Kings or some sort of \H A inquiry. 

  If showing the Joker takes you above 9NT, respond as if you are responding to a 9NT Specific Ace inquiry. 




\newpage
\section*{TK 6NT}

\todo[inline]{Thinking about removing this. Instead, 6NT opening should be a strong balance hand (i.e no good primary suit) and a 6NT response to partner's bid might be a strong balanced or a good raise (if a cue raise isn't available??).}

This is a system developed by me and Bazli (Tomlin-Karattiyattil) in 2015 when we first starting playing. The idea is to use NT bids when our side has at least 3 aces and the joker, then determine which aces we hold. Knowing about only 3 aces and the joker can help the cardplay and kitty decisions significantly easier.

6NT opening shows Joker and an Ace, and a not terrible hand. Responding 7NT shows either two Aces or the Joker, so that the partnership knows the have the Joker and 3 Aces between them. If opener holds Joker and an Ace, they now bid the Ace that they have. Responder responds with bidding the Ace that they don't have (only if possible at the 8 level). All 9 bids are natural. 8NT response to 7NT shows that opener has the missing Ace.

If opener holds 3 Aces after 6NT-7NT, they bid the Ace they don't have. Responder will know what's going on based on their holding \todo{is this cheating? Maybe in bridge}. 


\newpage
\chapter*{Carding}

See \textit{Notes on Play} for a summary of our carding agreements and an explanation of leading and signalling conventions.











\end{document}
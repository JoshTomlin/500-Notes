\documentclass[a4paper]{JoshCards}

\title{Notes on Play for 500}
\author{Josh Tomlin}
\pagestyle{fancy}
\date{}

\begin{document}

\maketitle

\chapter*{Summary of Agreements}

When trumps are being led, both players signal count (reverse) at first possibility (including if leading a low trump, following low to a trump, or if the opponents lead trumps). When a non-trump suit is played, we signal attitude (reverse) at first possibility. This includes when leading low, following low, and when the opponents lead a suit. Discards are attitude (reverse). Secondary suit preference signals apply whenever possible (though no one will notice)

We lead top of a sequence and top of nothing (including a doubleton). When following to a trick with equals, we play lowest of equals. Discarding an honour denies any higher honour and promises at least the next lower honour (unless from a singleton, say). It also carries the information that the discard could be afforded.

The meaning of all these leads and signals is explained in the next few pages.

\chapter*{Carding}

\section*{Signals}

When following suit with a low card, which low card should you play? Sometimes, it doesn't matter whether you play your 5 or 8 because they are never going to contend for a trick. Given this freedom, we have room to communicate with partner through the low cards we choose to play. Agreements to what our low cards mean is called signalling. It often goes unnoticed by the opposition, and it can be a great advantage to tell partner about your hand.

There are three types of basic signals: attitude, count and suit preference.

\subsection*{Attitude}

The most basic variation of attitude signals is known as \textit{high encourage}. That is, when following to a trick that you aren't attempting to win (e.g partner leads an offsuit Ace), or when discarding, a high spot card (spot card = not a picture card) tells partner that you like the suit, and a low card tells partner that you don't like the suit (or are indifferent about it). 

\todo[inline]{Examples}

It is important to notice that a `high' spot card is only relative to the cards you hold. If you hold AK65, then discarding the 6 is a encourage signal while discarding the 5 is a discouraging signal from your point of view. This isn't always clear from partner's point of view, the 6 looks pretty low to them. This can be clarified by a second discard. If you discard the 6 then the 5, partner will know that you played high first, then low, and can be sure that you have sent an encouraging signal. The same goes when you only have high spot cards, say J98. Discarding the 8 looks encouraging, but following it up with the 9 makes it clear that the 8 was a discourage. If you have A984 and want to encourage, don't play the 8 then the 9, because that is discouraging. Discard the 9, then the 4 if you wish (and can afford it). Of course, signalling should never override trick taking. If you have AQ4, don't discard the Q to encourage (that Q might be a trick!!). Instead, try discouraging other suits, and hopefully partner can work out what you have through a process of elimination.

Playing high encourage attitude signals is a good was to get used to thinking about signalling. It's natural because you only need to think about playing high when you want to send a signal (when not paying attention, people tend to discard their lowest, so partner won't get confusing or exciting signals). However, my preferred method of attitude signalling is \textit{low encourage}. That is, discarding or following with a low card is encouraging, and a high card is discouraging. This makes more sense because higher cards in a suit that I like are more valuable than low cards. It's important to notice what spot cards partner first discards, because the order that they discard can tell you what suit is more important to them. Further, after the initial couple discards, partner may not have anything else to say and will just be discarding cards they don't want. It is up to good judgement sometimes to figure what is an attitude signal and what isn't.

In bridge, signals are only useful on defence to give partner information about your hand (and therefore declarer's hand). In 500, it is useful to signal as declarer as well, to help partner play their hand well and find your honours.

\subsection*{Count}

Another signal that can be given through spot cards is count. A count signal tells partner whether you started with an even number of cards in the suit lead or an odd number. The method I like to play (and is pretty much standard in youth bridge) is called reverse count. This means that when following to a suit, playing a \textit{high} spot card shows an \textit{odd} number of cards in the suit (starting number of cards, always) and playing a \textit{low} spot card shows an \textit{even} number. There is the same ambiguity about what is high or low relative to partners hand, and this can be clarified by their second card played in the same way it can with an attitude signal. That is, following with a high card then a low card shows an odd number, and following low then high shows an even number. The way I remember this is: ``It would be \textit{odd} to play \textit{high}'', which is true. Using the information that partner has an odd or even number of cards in a suit, it is usually clear from context whether partner has 2 or 4, 3 or 5, etc.

\todo[inline]{Examples}

When a signal is count and when a signal is attitude is up to partnership agreement. I play that when following to a trump lead (while defending or declaring), we signal count (reverse count, as explained above). It is great when declaring to know exactly how many trumps partner has, which is not always clear from the bidding. It is also useful defending to know whether partner has a stray trump for a potential ruff or entry, or declarer has the rest. This is a massive help towards counting declarer's tricks on defensive. Knowing declarers trumps tells you about most of their hand, and helps figure out what to do offsuit.

On discards we signal attitude (low encourage). Knowing where partner's high cards are is useful information for setting up tricks, cashing tricks, or looking for places to discard losers (as declarer). When partner leads a suit, we give attitude (low encourage) for the suit being lead. Essentially telling partner whether or not we want the suit continued. For example, partner leads the A and you have Kxx, we encourage if we want to cash the K. On an opposition lead, it is not always clear what we signal (or if we signal at all). I tend to still signal attitude, in case the opponents are setting up tricks for me and I need to tell partner about this (which happens a lot). Although I have not discussed this agreement with any partners yet.



\section*{Following Suit}

When following suit, the carding is slightly different to when leading a suit. If holding AKxx and partner leads small to you, you almost always want to win the trick with the A or K. Since you hold both the A and the K, those cards are the \textit{same} value to you, because no one else can beat either one. We call cards that differ by one rank \textit{touching cards} or \textit{equals}. More examples of touching cards are AKQ, KQ, KQJ, QJ10, 9876, 87, etc.

When being lead to, if we choose to play a specific high card (high being higher than the currently played cards), we play the \textit{lowest of touching cards}. This is important when choosing to play an honour, In our example above, if partner leads to us holding AKxx, we win the K instead of winning the A. This (very standard) agreement has two advantages. One, now winning the A automatically denies the K (you would have played the K if holding AK). Two, if you win on the King, it looks likely that you also have the Ace. This is because if you win the King, that carries the additional information that the person after you did not beat your King with their Ace (if they had it) which they would do a lot of the time. This information is not available if you choose to win the Ace from AK. That being said, winning the King does not guarantee holding the Ace (partner could have it, second seat might have ducked if they are a good player), but it helps figure out the location of the missing Ace. This logical also applies to holdings such as KQxx, KQJx, AKQx, KJ10xx (if you choose to play the J or 10, then play the 10). Even small holdings such as 98xx, 109xxx, if you choose to play high.


\section*{Leading a Suit}

When leading a suit, we can choose to lead different cards based on what we hold in the suit. Our choices of what to lead are made with two things in mind: communicating to partner and not blowing tricks. 

When holding an honour sequence (e.g AKxx, KQx, KQJxx, QJ10xx) we lead \textit{top of a sequence}. This ensures that we are knocking out opponents honours to set up our own tricks, while telling partner about our suit quality. It is important to notice that this is opposite to when we are being led to (in which case we play the lower of equals). When partner leads an honour, you know that they 1. don't have the honour above that and 2. they have the honour below that (almost always). This can help you decide whether to finesse, duck or cover. For example, if you hold Axx and partner leads the Q, they have the J (and probably the 10) and your side is missing the K. Thus, if you need more offsuit tricks then taking the finesse (playing small instead of the Ace) can set up your offsuit tricks, and gains an extra trick when the K is trapped onside. When holding KQx(x), leading the K guarantees setting up atleast one trick, whereas leading low may lose to the J if your side is missing the A and J. Sometimes, it may be necessary to underlead the KQxx (if you know partner has the A or J, or you need them to have an honour in this suit). Thus these leading rules are only guidelines for when you are lacking enough information about the hand to make a more informative play.

\todo[inline]{Example, Overleading AKxx instead of leading the K situations (like ruffing)}

This also applies to interior sequences (e.g KJ10xx, or certain AJ10xx, AQJxx if you are not leading the A for whatever reason). In these suits, we lead \textit{top of an interior sequence}.

The most important place where leading agreements matter is in the trump suit. When leading trump, if we are going to lead an honour sequence, we follow the above rules. For example, holding the Joker and Right Bower we will lead the Joker if we choose to lead high (top of a sequence). Holding the Joker, Ace, King we will the Ace if we choose not to lead the Joker (top of an interior sequence). These agreements don't override your judgement. If you have an honour sequence in trump, but think you need to underlead it from context, then you should underlead it. But if you are going to lead a sequence, lead from the top. The most useful lead in the trump suit is the lead of an interior sequence. It gives partner an opportunity to produce a high card for you, take a finesse against their high card, or give you an option for a finesse on the way back (if the cover and lead a trump back).

\todo[inline]{Examples, underleading sequences}

In trumps, if we decide to lead low then we lead \textit{count} (specifically reverse count, as explained earlier). This usually tells partner how many trumps you hold, along with context from the bidding. For example, holding JoAQ7, lead the 7 since you have an even number (if you decided not to lead the Joker). From JoAQ85, lead the 8 (odd number). From JoAQ987, lead the 7 (even number). If partner sees you lead a low card, and then later you show up with a smaller or higher spot card (e.g you follow small or ruff small), then it will be clear whether you were showing an odd or even number. Sometimes it is not clear to partner how many trumps you had from the bidding (you might get some in the kitty too) so this lead helps clear that up. When returning the trump suit, we also lead count back if we decide to lead low (counting from your original holding of course).

If leading a doubleton (a two card suit) we lead \textit{top of a doubleton}. For example, from 10x lead the 10, from 8x lead the 8, even from something like Qx, you can usually lead the Q. From Kx, it's not so clear and you should use your judgement. Sometimes you underlead Kx, hoping the opponents hop Ace, and sometimes you lead the K hoping partner has the A or Q. This also applies when we win a  trick with honour third. For example, if partner leads a small trump to us and we hold Jo95, then we should win the Jo and return the 9 (\textit{top of a remaining doubleton}). This is essentially the same as a count signal (we return high instead of low, because we started with an odd number).

\todo[inline]{Examples}

The following offsuit leads are an idea I just had and has not been tested yet. I think it should work well, given the weird shortness of the offsuits in 500.

In bridge, it is often important to determine the distribution of the offsuits to take more tricks. In 500, there are so little cards offsuit (especially when in a suit contract) that distribution isn't as important. It's more important to know where the high cards are. When leading offsuit, we lead attitude (that is, low encourage as explained above). From a suit we like (with a good honour, say) we lead as low as possible. From a suit we don't like (10 high, Jxx, xxxx, say) we lead as high as we can afford. For instance, from K975, Q1065, K65, lead the 5 to tell partner that you like the suit. If for some magic reason you need to underlead A1084, lead the 4. From J94, 1096, 9865, lead the 9 to tell partner that you do not like the suit. This is consistent with leading top of a doubleton, because partner knows you can't an honour in that suit. Although sometimes partner will have to figure out whether you have a shortness and can give you a ruff, even if you've made a discouraging lead.


\section*{Advanced Signalling}


\subsection*{Suit Preference}

Suit preference is a much rarer and harder to read signal. It comes up in bridge, but I've yet to use it well in 500, because no one who I play with has learnt it yet.

Let's say you are playing with Hearts trumps and you hold A84 in some offsuit (let's say clubs). You draw trumps and through some count signal know that partner has two left. You lead the AC, to which partner discards. You would like partner to trump your two little clubs. So you could lead the 4C, they trump, but now what? You need them to give you the lead, so you can give them another ruff, but how? They can't lead trumps back, because that takes aways their trump, so they only have two options to lead back (in this case, diamonds or spades). This is where a suit preference signal helps partner get it right. For the first ruff, it doesn't matter whether you lead the 4 or the 8 to them. So leading the 4 asks them to return the lower ranked (rank = bidding order of the suit) suit, and leading the 8 for them to ruff shows preference for the higher ranked suit returned. In this case, leading the 8C asks for a Diamond return (Diamonds are higher than Spades in the bidding) and leading the 4C asks for a Spade return. The same idea applies when looking for ruffs on defence, a situation where it is more likely to be necessary to get it right on the first try (before declarer draws the rest of the trumps).\todo{Replace this with an actual hand example.}

The idea is, you can give partner a preference between two suits by whether you play a high spot card or a low spot card. It can only apply in situations where it is clear that there are only two options for suits, e.g the ruffing situation above, but not limited to that. In bridge, there is lots of room for subtle suit preference signals, and working these into a partnership can give you that little bit of extra luck in getting things right. 

Reading a suit preference signal is subtle and very situational, so it takes a lot of experience to know not only when to give a suit preference signal, but also when your partner might be giving you one. Here's an example from a game I played with Bazli last night where trusted signalling would have gotten us over the line in 10NT (a match decider!).

\begin{center}
Bazli\\
\hand*!{Y AKQJ}{A8}{Q}{A7}
    
Josh\\
\hand*!{8}{T654}{AK7}{QT}
\end{center}
\todo[inline]{Ideally I want to reprogrm the one-down package to accomodate for weird 500 rules/symbols. For now, I've got B/b = right/left bower and Y = Joker (which is terrible, would prefer a fancy J, but hey at least I can call it the Yoker).}

Oh this hand, Bazli bid 10NT over West's Open Misere (with no voluntary bidding from me). Looking at my hand with AK of diamonds, I'm feeling pretty confident. He's probably got a running suit (my guess is in spades), definitely has the Joker, and I've got a bonus two tricks for him to hopefully get us over the line.

Immediately he leads the JS, which for sure is going to hold and tells me he has at least top four spades. He starts running spades, and now I need to signal to tell him where my tricks are. I don't want to pitch down to AK tight in diamonds, and I don't want to give him a possibly confusing 7D discard (in retrospect this may be fine, by what if I had AK9 or AKT?). The 7D might even set up as a trick in this situation, since the defenders are under a lot of squeeze pressure. I want to keep my QT doubleton club to keep the clubs threatened and make the defence find awkward discards, so my one priority here was to discard hearts. I wanted to send a strong signal that I do not like hearts, so I pitched the T, followed by the 6 then the 5. Then I pitched the 7D to try to tell Bazli that I liked diamonds, but this wasn't so clear to read. I think it should be very clear that I don't like hearts, but how does he know what to do between clubs and diamonds?

This is a perfect situation for giving a suit preference signal when I discard hearts. On my second heart discard (after discarding the T), there is no difference between whether I discard the 6,5 or 4. If it wasn't already clear that I didn't like hearts, discarding any of these cards will clarify. However, I plan to discard all of these, so I have a choice in what order I discard them. So I should discard the 6, then the 4 (or 5) to tell partner that I like the higher ranked suit. That is, I like diamonds. Doing this allows me to encourage diamonds without needing to discard the 7D, a potentially useful card which is certainly difficult to read. If instead I had discarded the 4, then the 5 (or 6), this should tell partner about a Club preference (or possibly no preference). If I really wanted a Club, I would discard the 4, then the 6, then the 5, just to make sure partner knows that something is up.

On the actual hand Bazli cashed all his spades, cashed his two aces (on which East threw the KC, surprisingly setting up my QC, maybe it was a singleton), then tanked. He lead the Joker as a Heart, then exited a Heart playing me for the QH (I believe the KH either dropped or was already discarded). In this case, either a Club or a Diamond would have sufficed, and we would have won the match. Instead, we went down one and now I have no trickster chips.















\end{document}